\documentclass[12pt]{article}

% Any percent sign marks a comment to the end of the line

% Every latex document starts with a documentclass declaration like this
% The option dvips allows for graphics, 12pt is the font size, and article
%   is the style




% array/table
\usepackage{array}
\usepackage{multirow}

% figure
\usepackage[pdftex]{graphicx}
\usepackage{epsfig}
\usepackage[hang]{subfigure}
\usepackage[small,bf]{caption}
\usepackage{amsmath}
\usepackage{enumitem}
\usepackage{tabularx}
\usepackage{ctable}


\usepackage{authblk}
\renewcommand\Authands{ and }

%clickable ref
\usepackage[backref,pagebackref,naturalnames=true]{hyperref}



\setlength{\oddsidemargin}{0.25in}
\setlength{\textwidth}{6.5in}
\setlength{\topmargin}{0in}
\setlength{\textheight}{8.5in}



%----------------------------------------------------------------------------------------
%        DOCUMENT INFORMATION
%----------------------------------------------------------------------------------------

\title{EG01-EG23 Transition: Cyclus Results}


\author[1]{B. Mouginot\thanks{\href{mailto:mouginot@wisc.edu}{mouginot@wisc.edu}}}
\author[1]{P.P.H. Wilson \thanks{\href{mailto:paul.wilson@wisc.edu}{paul.wilson@wisc.edu}}}
\author[1]{R. Carlsen \thanks{\href{mailto:rcarlsen@wisc.edu}{rcarlsen@wisc.edu}}}
\affil[1]{University of Wisconsin--Madison, Department of Engineering Physics, CNERG group}


\date{\today}

\setlength{\parindent}{0em}
\setlength{\parskip}{0.7em}

\begin{document}
\maketitle

\section{Intro/specification}

One of the activities of the U.S. Department of Energy (DOE) Fuel Cycle Option (FCO) campaign is
to assess the future of the U.S. nuclear fleet. During the Evaluation and Screening study, 
several alternative fuel cycles were analyzed in an equilibrium state and organized into Evaluation
Groups (EG).  In the current phase the DOE has been interested in assessing a transition from the
current once-through fuel cycle to some of the best alternatives from their
equilibrium analysis.

The analysis in this report was conducted done after similar analyses performed by the
three fuel cycle simulator codes DYMOND, \cite{dymon}, VISION \cite{vision},
and ORION \cite{orion}. The purpose of this report is to demonstrate the
capabilities of the Cyclus simulator \cite{cyclus} as sufficiently mature to
contribute to ongoing fuel cycle analysis work.

\section{Scenario Description}

The scenario used in this study models the transition from a once-through
light water reactor (LWR) fuel cycle (i.e. EG01) to a 
full recycle case (i.e. EG23) with only sodium fast reactors
(SFR).  The LWR and SFR deployments were
selected to accomplish three primary goals:

\begin{itemize}
    \item Follow the 1\% annual electricity production curve (Figure \ref{fig:nrg}).
    \item Minimize the amount of sitting separated plutonium.
    \item Transition as quickly as possible given the other goals.
\end{itemize}

The rector deployment schedule (Figure \ref{fig:deploy}) is part of the problem
specification. It has been taken from a DYMOND calculation by B. Feng\cite{B.Feng_calculation} 
and implemented in
Cyclus (with some approximation which will be detailed further in the
document).

Other scenario details, including how to represent the problem specifications in Cyclus,
are based on previous work done by R. Carlsen
(https://github.com/rwcarlsen/eg23-sim).  From this study, the configuration
and deployment schedule of the separation facilities for LWR used fuel have
been taken.

\begin{figure}[h!]
    \centering
    \subfigure[Deployment schedule\label{fig:deploy}]  {\epsfig{figure=img/CapacityStarted,width=0.48\textwidth}}
    \subfigure[Electricity produced\label{fig:nrg}]    {\epsfig{figure=img/ElectricityGenerated,width=0.48\textwidth}}
    \caption{
        Deployment schedule and corresponding electricity produced by the
        different reactors where R1(blue): existing LWR, R2(green): new
        builded LWR, R3(red): high SFR, R4(purple): low
        SFR.\label{fig:deployment}
    }
\end{figure}

By default, Cyclus facilities exhibit greedy behavior: they request and
process as much material as they can given their inventory capacities and
process throughput limits. As Cyclus does not yet include direct support for on-demand
processing, some special accommodations were made to maintain the desired low
inventory of separated Pu. In order to address this in parts, a series of two
simulations were run:

\begin{itemize}

    \item \textbf{Case 1}: A simulation with fuel fabrication deployments
        matching capacity for SFR fresh fuel demand over time, but with
        unrestricted, greedy spent fuel separations.

    \item \textbf{Case 2}: A simulation with separations throughput changing
        over time. Results from the first simulation were used to calculate
        the required separations deployments over time in order to approximate
        on-demand separations.

\end{itemize}

Scenario details for cases 1 and 2 are identical except for the changes noted in
Section \ref{sec:case2}.

The following subsections describe the configuration of the primary facility
types (i.e. Cyclus prototypes) used in this analysis.

\subsection{Reactors}

After the LWR to SFR transition is completed, new SFR reactors are deployed
with a lower breeding ratio in order to prevent unbounded buildup of
plutonium. The properties of the different reactors cores used for the
simulation have been summarized in Table \ref{tab:reactor}. Two different SFR 
configurations exist to facilitate management of Pu in later years by reducing
the breeding ratio.  Because of the 1-month timestep of Cyclus, the batch quantity and cycle length
parameters have been manipulated to match as closely as possible important
invariants such as fuel residence time, burnup, and discharge rates
\cite{B.Feng_calculation}. The notion of batches
is defined in Cyclus by combining the number of assemblies per core and 
the number of assemblies per batch.  In the simplest form, an integer number of
batches, n, can be defined by specifying n assemblies per core and 1 assembly
per batch.  For a non-integer number of batches, f, can be implemented by
specifying f$\times 10^m$ number of assemblies per core and $10^m$ assemblies
per batch, where f$\times 10^m$ is an integer.  For this problem, it is also
possible to select the cycle length, number of batches and batch size to
approximate the same fuel throughput.  This choice implies small 
discrepancies between specifications and Cyclus input that have been detailed 
below.

\begin{table}[h!]
    \centering
    \begin{tabularx}{350pt}{lXXX}
      \hline
      Core Specifications   &  LWR             &  SFR 1            &  SFR 2            \\
      \hline
      Rated Power, [MWe]    &  1000            &  400              &  400              \\
      Thermal Efficiency    &  0.33            &  0.4              &  0.4              \\
      Capacity Factor       &  0.90            &  0.90             &  0.90             \\
      Burnup, [Gwd/tHM]     &  50              &  47.501           &  42.397           \\
      Number of batches     &  3               &  5.44             &  4.44             \\
      Cycle length, [month] &  18              &  12               &  12               \\
      Core Inventory, [tHM] &  3$\times$29.565 &  37.62            &  34.40            \\
                                                                                       \\
      \multicolumn{4}{l}{Cyclus implementation}                                        \\ 
      \hline
      \hline
      Assembly size [kgHM]  &  29565           &  1383.09          & 1549.55           \\
      Assembly in the core  &  3               &  27               &  22               \\
      Assembly per batches  &  1               &  5                &  5                \\
      Cycle length, [month] &  18              &  12               &  12               \\
      Core Inventory, [tHM] &  3$\times$29.565 &  27$\times$1.383  &  22$\times$1.549  \\
      \hline
    \end{tabularx}
    \caption{Reactor core properties.}
    \label{tab:reactor}
\end{table}


For the SFR 1 facility type, the number of batches is specified to be $5.44$
(for discharge burnup of 37.62 GWd/t), $5.40$ has been used (27/5). For the SFR
2, $4.4$ batches (22/5) have been used to approximate the 4.44 batches that were
specified.  In order to maintain the material flow in and out of the SFR reactor
a small correction have been made on the reactor core size by a factor
$5.4/5.44$ for the SFR A and $4.4/4.44$ for the SFR B. This does not affect the
composition of the discharge fuel as the reactor model used for this analysis
does not perform a burnup evolution calculation.

An exact match to the specified number of batches is possible, but
inefficient.  For example, SFR 1 could have been implemented with
$544~$assemblies per core and $100~$assemblies per batch, but this would
result in 20 times as many material objects being traded by the dynamic
resource exchange (DRE), and thus longer runtimes and larger output databases.

All LWR reactors use enriched uranium fuel. The SFR used MOX fuel with
different plutonium enrichments. The input/output fuel recipes are summarized
in Table \ref{tab:reactor_fuel} taken from \cite{B.Feng_calculation}.

\begin{table}[h!]
    \centering
    \begin{tabular}{lllll}
    \hline
    \multicolumn{2}{c}{Reactor}            &  LWR [$\%w$] &  SFR fuel 1 [$\%w$] &  SFR fuel 2 [$\%w$] \\
    \hline
    \multirow{3}{*} {In recipe}  &  235U   &  95.8        &  0                  &  0                  \\
                                 &  238U   &  4.2         &  92.36              &  91.466             \\
                                 &  Pu     &  0           &  7.64               &  8.534              \\
    \hline
    \multirow{5}{*} {Out recipe} &  235U   &  0.8         &  0                  &  0                  \\
                                 &  238U/U &  92.68       &  85.99              &  86.025             \\
                                 &  Pu     &  1.2         &  9.02               &  9.596              \\
                                 &  MA     &  0.11        &  0.13               &  0.107              \\
                                 &  FP     &  5.21        &  4.86               &  4.272              \\
    \hline
    \end{tabular}

    \caption{
        Input/Output Fuel composition recipe for the different reactors. Note that
        for the SFR reactor fuel no isotopic distinctions have been made and U in
        SFR should be considered depleted uranium in the input recipes, the
        uranium isotopic changes in the output recipes have not been investigated
        in this work.
    }

    \label{tab:reactor_fuel}
\end{table}

\subsection{Cooling/Storage}

After irradiation, all fuel is cooled for 84 months before becoming available
for reprocessing.  The Cycamore Storage facility allows defining a minimum
residence time for each incoming material.  In order to 
differentiate between material in cooling and material available for
reprocessing as specified in the reporting requirements, 
2 storage facilities are used in series providing this explicit
distinction. One dedicated to cooling, with a residence time of 84 months,
corresponds to the output quantity ``Used Fuel in Cooling Storage'', and
one dedicated to storage until reprocessing with no minimum residence time
corresponds to the output quantity ``UNF waiting for reprocessing''.

The implementation properties of the cooling storage can be found in table \ref{tab:cooling_1} 
\begin{table}[h!]
    \centering
    \begin{tabular}{ll}
    \hline
    \multicolumn{2}{l}{Cooling Implementation}  \\
    \hline
    Residence Time [month]   &  84  \\
    \hline
    \end{tabular}
    \caption{Cooling facilities core implementation.}
    \label{tab:cooling_1}
\end{table}

\subsection{Separation}

\subsubsection{Case 1: Greedy Separations}

The LWR fuel separation is handled by three identical separation facilities, two
deployed in 2030 and one in 2040. In Case 1, the SFR separation facilities have
an effectively infinite separation capacity, but the separation rate has been
throttled indirectly using the gradual fuel fabrication capacity growth. 
As the separation output buffers are not limited, those separation facilities will
reprocess all the used available fuel with a constant reprocessing rate.

As the fuel fabrication facilities have a tuned deployment and capacities (for
10 batches of fuel), the separation facilities' output buffers are the basis for
the ``Fuel Resource in Storage'' result.  The reprocessed uranium is sent after
the separation to a dedicated storage.

Notable are subtle differences in what these two storage facilities data represent
and their exact semantic meaning in the analysis spreadsheet.

\begin{table}[h!]
    \centering
    \begin{tabular}{lllll}
    \hline
    Separation Implementation  &  LWR        &  SFR 1/2    \\
    \hline
    Throughput [tHM/month]     &  83.3333    &  5000       \\
    feed buffer [tHM]          &  107.537    &  5000       \\
    Pu output buffer [tHM]     &  Unlimited  &  5000       \\
    Pu separation efficiency   &  0.99       &  0.99       \\
    Recycled U buffer [tHM]    &  Unlimited  &  Unlimited  \\
    U separation efficiency    &  0.99       &  0.99       \\
    Waste buffer [tHM]         &  Unlimited  &  Unlimited  \\
    \hline
    \end{tabular}
    \caption{Separation facilities core properties. }
    \label{tab:separation_1}
\end{table}

\subsubsection{Case 2: On-Demand Separations}
\label{sec:case2}

The only difference from the Case 1 scenario is the way in which reprocessing
capacity is handled over time.  Modeling on-demand separations is approximated by
deploying many small separations facilities according to plutonium requirements
for SFR fuel fabrication.  These Pu requirements were determined from the Case 1
results. While the calculated deployment schedule does not follow plutonium usage
exactly, it is a reasonable approximation.

For this calculation, all of the reprocessing facilities are based on only
facility prototype.  When facilities based on this prototype request used fuel
from reactors, they prefer irradiated SFR fuel over UOX fuel. When thre is not
preferences difference on the used UOX fuel origin (LWR A or B), those
reprocessing facilities prefer used fuel from the lower breeding ratio SFRs
over used fuel from the higher breeding ration SFRs. The implemented
characteristics of each of the reprocessing facilities used in Case 2 are
summarized in Table \ref{tab:fuelfab_2}.

\begin{table}[h!]
    \centering
    \begin{tabular}{ll}
    \hline
    Separation Implementation  &  All Fuel   \\
    \hline
    Throughput [tHM/month]     &  60         \\
    Feed buffer [tHM]          &  66         \\
    Pu output buffer  [tHM]    &  6          \\
    Pu separation efficiency   &  0.99       \\
    Recycled U buffer [tHM]    &  Unlimited  \\
    U separation efficiency    &  0.99       \\
    Waste buffer [tHM]         &  Unlimited  \\ 
                                             \\
    \multicolumn{2}{c}{Feed commodities}    \\
    \hline
    Commodity Sources         &  Preferences \\
    LWR UOX A                 &  3           \\
    LWR UOX B                 &  3           \\
    SFR MOX A                 &  2           \\
    SFR MOX B                 &  1           \\ 
    \hline
    \end{tabular}
    \caption{Separation facilities implementation.}
    \label{tab:fuelfab_2}
\end{table}

The Cycamore separation facility requests the feed commodities, accordingly to the
set preferences, to fill its feed buffer. The material in the feed buffer is then
separated following the implemented separation instruction and efficiency all the
unused material is send to the waste. The reprocessing rate is limited by:

\begin{itemize}

  \item the feed buffer inventory
  \item the total throughput which limits the amount of material reprocessed from
    the feed buffer,
  \item the output stream buffer size.

\end{itemize}
If one of these parameters reaches its limit the separation is stopped until the
constraint is released, either by the occurrence of the next timestep (for the
throughput limits), or by a change in the buffer's inventory. 

\subsection{Fuel Fabrication}

The UOX fuel fabrication is handled by one enrichment facility, the properties
of this enrichment facilities are summarized in Table \ref{tab:enrich_1}. As
explained above, the specification of the enrichment facility has been taken from the
R. Carlsen calculation and might differ from the exact EG01-23 transition
specifications. 

\begin{table}[h!]
    \centering
    \begin{tabular}{lllll}
    \hline
    Enrichment implementation  &  UOX        \\
    \hline
    Throughput [tHM/month]     &  Unlimited  \\
    SWU capacity [tSWU/month]  &  Unlimited  \\
    Tails assay                &  0.0025     \\
    Feed buffer [tHM]          &  Unlimited  \\
    \hline
    \end{tabular}
    \caption{Enrichment facilities implementation. }
    \label{tab:enrich_1}
\end{table}

Fuel fabrication for other fuels is implemented with the Cycamore fuel
fabrication archetype that mixes two input streams according to the
plutonium-equivalent theory\cite{Pueq}. A reactivity weight is calculated for
the required output composition and both of the input streams. An interpolation
is used to compute a mixing ratio of the input streams that matches the target
weight.


The SFR fuel fabrication facilities are deployed as a rate of 1 for every 10
SFRs.  As the fuel composition and annual material flow change slightly between
SFR 1 and 2, the specifications between fabrication facilities are also slightly
different. The details of all fuel fabrication facility characteristics are
summarized in Table \ref{tab:fuelfab_1}.

\begin{table}[h!]
    \centering
    \begin{tabular}{lllll}
    \hline
    Fuel Fab Properties      &  SFR 1   &  SFR 2   \\
    \hline
    Throughput [tHM/month]   &  75.240  &  76.440  \\
    Depleted U buffer [tHM]  &  69.492  &  69.912  \\
    Pu buffer  [tHM]         &  5.748   &  5.856   \\
    \hline
    \end{tabular}
    \caption{Fuel fabrication facilities properties.}
    \label{tab:fuelfab_1}
\end{table}


\section{Results}

\subsection{Case 1: Greedy Separations}

All fuel loading metrics (Figure \ref{fig:ResourceUsed}) are the same or very
close to DYMOND simulation results. There are some observable fluctuations on the
annual fuel loading and the SWU requirements which are consistent with the
synchronized refueling cycles of the initially deployed reactors.  There are
various simple ways to stagger refueling cycles for smoother data that were not
pursued in this analysis.  Similar fluctuations appear as ``echoes'' in other
material flows later in time.

\begin{figure}[h!]
    \centering
    \subfigure[Resources Mined]          {\epsfig{figure=img/ResourceMined,width=0.48\textwidth}}
    \subfigure[SWU Requirement]           {\epsfig{figure=img/SWURequierment,width=0.48\textwidth}}
    \subfigure[Annual Fuel Loading Rate]  {\epsfig{figure=img/AnnualFuelLoading,width=0.48\textwidth}}
    \caption{Resources production and fuel loading.\label{fig:ResourceUsed} }
\end{figure}

The generated power and the deployment schedule correspond as expected to the
required one (shown on Figures \ref{fig:deploy} and \ref{fig:power} ) match
perfectly with results from the DYMOND simulations. The time scale in the output
spreadsheet filled starts in 2015 and ends in 2215, while the DYMOND calculation
used as reference for the deployment schedule ends in 2210. The sudden drop in
2210 is due to a lack of deployment data after 2210.


The other curves also appear as expected. The annual reprocessing rate (Figure
\ref{fig:ARR_1}) corresponds directly to the deployments. Spent UOX
reprocessing starts in 2010 with 2000 MTHM/yr with another 1000 MTHM/yr added in
2030. This reprocessing rate (UOX1 + UOX2) continues until the end of the
production of used UOX (i.e. when all LWRs are decommissioned).



MOX spent fuel production by the 2 different SFR types roughly follows the
loading of fresh fuel.  Because the reprocessing facilities are greedy,
reprocessing all used fuel available, there is no SFR spent fuel waiting for reprocessing
because it is directly reprocessed after cooling.

The amounts of the different fuels in ``Used Fuel in Cooling Storage'' (Figure
\ref{fig:UFCS_1}) follow the discharge of the different reactors as expected.


Almost all SFR spent fuel is directly reprocessed after cooling. The
reprocessing capacity of SFR fuel are large enough to prevent any SFR fuel
accumulation in ``UNF Waiting for Reprocessing'' (see Figure \ref{fig:FWR_1}).

The accumulation of LWR used fuel in ``UNF Waiting for Reprocessing'' is induced
by the limited constant reprocessing capacity for that fuel. The amount of UOX
spent fuel in storage stops increasing when the production rate of used UOX fuel
becomes smaller than its reprocessing rate, i.e. when the production of used
fuel drops below $2\times83.3333$ tons.
 


Before the start of the first FBR, the reprocessing capacities of the LWR fuel
is higher than the demand on plutonium to fabricate the MOX fuel, creating an
accumulation of plutonium prior to 2100, as shown in the Figure
\ref{fig:FIS_1}. After 2100, with the decommissioning of the LWRs and the
consequent expansion of SFRs, less LWRs used fuel is produced, allowing the
reduction of the inventory of plutonium. The reprocessing capacities of the FBR
used fuel are well adapted to the demand, but the plutonium storage experiences
a small accumulation just before the SFRs switch to lower breeding ratio.
 

\subsection{Case 2: On-demand Separations}

The limitation added on the reprocessing capacities( Figure \ref{fig:ARR_2}) allows to limit the
inventory of separated plutonium. Indeed, as seen in Figure \ref{fig:ARR_2}, the
reprocessing follows more closely than in case 1 the plutonium need of the SFR
fuel fabrication: the accumulation of plutonium is limited to $\sim500~t$
(Figure \ref{fig:FIS_2}) versus $\sim1500~t$ in the first case. Nevertheless,
one can still observe a large plutonium accumulation spike 2100 and 2145 that
could be avoided with a better reprocessing capacity deployment.

\begin{figure}[h!]
    \centering
    \subfigure[Case 1.\label{fig:ARR_1}] {\epsfig{figure=img/AnnualReprocessingRate_1,width=0.48\textwidth}}
    \subfigure[Case 1.\label{fig:ARR_2}] {\epsfig{figure=img/AnnualReprocessingRate_2,width=0.48\textwidth}}
    \caption{Annual Reprocessing Rate.}
    \label{fig:ARR_1}
\end{figure}
\begin{figure}[h!]
    \centering
    \subfigure[Case 1.\label{fig:FWR_1}] {\epsfig{figure=img/UNFWaitingReprocessing_1,width=0.48\textwidth}}
    \subfigure[Case 2.\label{fig:FWR_2}] {\epsfig{figure=img/UNFWaitingReprocessing_2,width=0.48\textwidth}}
    \caption{Fuel waiting for Reprocessing.\label{fig:ARR_FWR_SFC} }
\end{figure}
\begin{figure}[h!]
    \centering
    \subfigure[Case 1.\label{fig:ARR_1}]   {\epsfig{figure=img/AnnualReprocessingRate_1,width=0.48\textwidth}}
    \subfigure[Case 2.\label{fig:ARR_2}]   {\epsfig{figure=img/AnnualReprocessingRate_2,width=0.48\textwidth}}
    \caption{Annual Reprocessing Rate.\label{fig:ARR_FWR_SFC_2} }
\end{figure}
\begin{figure}[h!]
    \centering
    \subfigure[Case 1.\label{fig:FIS_1}]  {\epsfig{figure=img/FuelInStorage_1,width=0.48\textwidth}}
    \subfigure[Case 2.\label{fig:FIS_2}]  {\epsfig{figure=img/FuelInStorage_2,width=0.48\textwidth}}
    \caption{Fuel in Storage.\label{fig:ARR_FWR_SFC_2} }
\end{figure}
Because of both the reprocessing priority and "on-demand" separations
deployment, more LWR spent fuel accumulates waiting for reprocessing with a quick drop
starting around year 2090 as stored up plutonium is used in the first wave of
SFR deployments replacing aging LWRs (see Figure \ref{fig:FWR_2}). Some of SFR fuel also accumulates
waiting for reprocessing, but it eventually decreases, showing a well designed SFR
deployment schedule that reaches an comfortable equilibrium generating
plutonium at the same rate it is being recycled.

\section{Summary}

This study has shown the capability of Cyclus to properly simulate the EG01 to
EG23 fuel cycle transition.   

Although Cyclus is not currently able to simulate on-demand behavior directly,
it is possible to deploy the different facilities with limited capacities to
"follow" material demand to mimicking an on-demand type of behavior, as shown in
Case 2. There are some small differences in the pattern of fuel
loading (and almost all the reactor fuel metrics), deriving from the way
batches are modeled discretely in Cyclus while in DYMOND all 
material flows are managed in a more continuous way.

The main difficulty encounter during this study was to figures out the meaning
of the output variable such as ``Fuel Resource in Storage'', ``UNF waiting for
reprocessing'', as well as the conversion of the input specification into proper
cyclus input parameter. Moreover approximations have been to reduce the
computing effort, those approximation could have been avoided if a precise
simulation is required.


%----------------------------------------------------------------------------------------
%        BIBLIOGRAPHY
%----------------------------------------------------------------------------------------

\bibliographystyle{unsrt}

\bibliography{EG23}

%----------------------------------------------------------------------------------------

\end{document}


%%% Local Variables:
%%% mode: latex
%%% TeX-master: t
%%% End:
