\documentclass[12pt]{article}

% Any percent sign marks a comment to the end of the line

% Every latex document starts with a documentclass declaration like this
% The option dvips allows for graphics, 12pt is the font size, and article
%   is the style

\usepackage[pdftex]{graphicx}
\usepackage{url}

% These are additional packages for "pdflatex", graphics, and to include
% hyperlinks inside a document.

\usepackage{authblk}
\usepackage[T1]{fontenc}
\usepackage[utf8]{inputenc}
\usepackage[backref,pagebackref,naturalnames=true,colorlinks]{hyperref}

\renewcommand\Authands{ and }


\setlength{\oddsidemargin}{0.25in}
\setlength{\textwidth}{6.5in}
\setlength{\topmargin}{0in}
\setlength{\textheight}{8.5in}

%----------------------------------------------------------------------------------------
%	DOCUMENT INFORMATION
%----------------------------------------------------------------------------------------

\title{EG01-23: CYCLUS Calculation Report}


\author[1]{B. MOUGINOT\thanks{\href{mailto:mouginot@wisc.edu}{mouginot@wisc.edu}}}
\author[1]{P.P.H. WILSON \thanks{\href{mailto:paul.wilson@wisc.edu}{paul.wilson@wisc.edu}}} 
\author[1]{R. CARLSEN \thanks{\href{mailto:carlsen@wisc.edu}{carlsen@wisc.edu}}} 
\affil[1]{University of Wisconsin-Madison, Department of Engineering Physics, CNERG group}


\date{\today}

\begin{document}
\maketitle 





\section{Intro/specification}
The goal of this study is to model the transition from once-through PWRs (EG01) to full recycle FBRs (EG23). It appears that the immediate transition between the existing PWR fleet (EG01) and a full FBR fleet (EG23) was not directly possible. In order to have enough plutonium inventories to do the transition, a few LWRs must be deployed before starting to replace LWRs with FBRs. The deployment of the new LWRs and the FBRs was calculated to:
\begin{itemize}
\item minimize the amount of plutonium in the cycle
\item follow the energy production requested (Fig 1b.).
\end{itemize}
The deployment schedule (Fig 1a.) has not been calculated using CYCLUS, it has been taken from B. Feng calculation and implemented in CYCLUS (with some approximation which will be detailed further in the document).

 	Fig. 1a. Deployment schedule				Fig. 1b.Electricity produced
Fig. 1. Deployment schedule and corresponding electricity produced by the different reactors where
 R1(blue): existing PWR, R2(green): new builded PWR, R3(red): high FBR, R4(purple): low FBR 



\section{Prototype configuration: what \& why}
This calculation is based on previous calculation operated by R. Carlsen, which can be found there. From this study, the configuration and deployment schedule of the separation facilities for PWR used fuel has been taken. The rest of the study has been adapted from it, to match B. Feng DYMOND calculation \cite{}.

As Cyclus does not yet include direct support for on-demand processing, the separation and the fabrication have been modeled as follows to approximate on-demand separation, thus minimizing inventories of separated plutonium.
To achieve this, two differents calculations have been made. The first one correspond the modelisation of a growing MOX fuel fabrication capacity following the MOX fuel requirement. The second one models both MOX fuel fabrication and fuel separation with limited capacities, which discreetly grow as the demand grows.

In both the reactor characteristic (core properties, fuel recipes, deployment schedule) are the same. 
\subsection{Reactors}
After the PWR/FBR transition, new FBR reactor are deployed with a lower breeding ratio, in order to minimize the plutonium inventory. The properties of the differents reactor core used for the simulation have been summarized in Tab. 1.

Core Properties
PWR 1 \& 2
FBR 1
FBR 2
Rated Power, [MWe]
1000
400
400
Thermal Efficiency
N.A.
0.4
0.4
Capacity Factor
0.90
0.90
0.90
Number of batches
6
5
3
Cycle length, [month]
9
13
18
Core Inventory, [tHM]
14.784*6
7.524*5
11.466*3
Tab. 1. Reactor core properties.

Note that, because of the 1 month timestep of CYCLUS, some liberty has been taken with the coupled Batch-Quantity/Cycle-length in order to match as closely as possible with the specified quantities [REF-F.Bo]. The batches have been sized accordingly. All PWR reactors use enriched uranium fuel. The FBR used MOX fuel with different plutonium enrichments. The input/output fuel recipe are summarized in the Tab.2 [REF-F.Bo].


Reactor
PWR [%w]
FBR fuel 1  [%w]
FBR fuel 2  [%w]
In recipe
235U
95.8
0
0
238U
4.2
92.36
91.466
Pu
0
7.64
8.534
Out recipe
235U
0.8
0
0
238U/U
92.68
85.99
86.025
Pu
1.2
9.02
9.596
MA
0.11
0.13
0.107
FP
5.21
4.86
4.272
Tab. 2. Input/Output Fuel composition recipe for the different reactors. Note that for the FBR reactor fuel no isotopic distinctions have been made and U in FBR should be considered depleted uranium in the input recipes, the uranium isotopic change in the output recipes have not been investigated in this work.  

\subsection{Cooling/Storage}
After Irradiation, all fuel is sent to dedicated cooling storage, then they are transferred to a longer term storage after 84 month.
\subsection{Separation}

The LWR fuel separation is handled by three identical separation facilities, two deployed in 2030 and one in 2040.The FBR separation facilities have a very large separation capacity, in the first case, we have tried to limit the separation flux using the fuel fabrication facilities: since the fabrication if full as well as the separation, it will not request any more plutonium, so the separation will stop the separation?


Separation Properties
PWR 
FBR 1/2
Throughput [tML]
83.3333333
5000
feed buffer [tML]
107.537
5000
Pu output  [tML]
Unlimited
5000
Pu separation efficiency
0.99
0.99
Recycled Uranium [tML]
Unlimited
Unlimited
U separation efficiency
0.99
0.99
Waste [tML]
Unlimited
Unlimited
Tab. 3. Separation facilities core properties.


\subsection{Fuel Fabrication}
The UOX fuel fabrication is handled by one enrichment facility, the properties of this enrichment facilities are summed up in Tab. 4. 

Enrichment Properties
UOX
Throughput [tML]
Unlimited
swu capacity [tML]
1e97
tails assay  
0.0025
Initial feed [tML]
Unlimited
Tab. 4. Enrichment facilities properties.

The FBR fuel fabrication are suppose to handle and been deployed as a rate of 1 for 10 FBR reactors. Since the fuel composition and annual flux slightly change between FBR 1 and 2,, the specifications between fabrication are, as well, slightly differents... The detail of all fuel fabrication facilities characteristics can be found in Tab. 5.


Fuel Fab Properties
FBR 1
FBR 2
Throughput [tML]
75.240
76.440
depleted buffer [tML]
69.492
69.912
Pu buffer  [tML]
5.748
5.856
Tab. 5. Fuel fabrication facilities properties.


\subsection{Results}
All fuel loading metrics (Fig. 2) are the same or very close to the VISION simulations. Nevertheless one can observe some fluctuation on the annual fuel loading and the SWU requirement which are consistent with the batching of the initially deployed reactors which are synchronized.  


Fig. 2a. Ressources Mined				Fig. 2b SWU Requirement		 

Fig. 2c. Annual Fuel Loading Rate


The generated power and the deployment schedule (Fig. 3)  match perfectly the one of the VISION simulations. The sudden drop in 2210 is due to a lack of data after 2210: no new reactor have been started after this date.

Fig. 3a. Electricity Generated				Fig. 3b Capacity Started		

The other curves are as expected. Indeed, the annual reprocessing rate correspond to the deployment :  for the UOX fuel start the reprocessing in 2010 with 2/3 of the capacity, and then an increase of 1/3 in 2030, and then a constant reprocessing rate (UOX1 + UOX2) until the end of the production of used UOX (all PWR decommissioning).


 
Fig. 4. Annual Reprocessing Rate


 For the MOX fuel produced by the 2 different FBR types, it follows the loading of fresh fuel, with some shift : the reprocessing facility are greedy : they reprocess all the used fuel available. In consequence there is no FBR used fuel in storage they are directly reprocess after the cooling.




Fig. 5a. Used FUel in Cooling			Fig. 5b Fuel waiting for reprocessing



Concerning the fuel in storage, as almost all FBR used fuel are directly reprocess after cooling, all quantity barely stop increasing after all the UOX fuel have been reprocess. 

Fig. 6. Storage Fuel composition

It appears that the capacity of the FBR fuel reprocessing are not enough to reprocess the full income of MOX fuel until 2195: this explain the some appearance of MOX fuel in storage waiting for reprocessing. With the deployment of the low breeder FBR, the high breeder reactor are decommissioned decreasing the incoming irradiated MOX fuel,  the reprocessing capacity is again higher than the sued MOX (type 1) production...


As shown in the figure 7a \& 7b, the quantity of plutonium in the storage follow the variation of the amount of MOX1 fuel in the storage.
	

Fig. 7a. Storage Fuel composition (zoom)		Fig. 7b Fuel waiting for reprocessing (zoom)		 

\section{On demand Mimic Calculation}
In parallel of the previous calculation, I have tried to perform a calculation, where the separation facilities have limited capacities and are deployed accordingly to the plutonium requirement for MOX fuel fabrication. This deployment schedule is not perfect, but it provides a good idea to how should evolve such kind of calculation? 
The only difference with the previous calculation is the way to define the reprocessing.

For this calculation, all fuel are reprocessed by only one kind of reprocessing facility. The fuel input preferences has been set to 3 to all irradiated UOX fuel, 2 for MOX fuel irradiated in high breeder, and 3 for the low breeded fuel (higher the number is higher the preferences is). The characteristic of those reprocessing facilities are summed up in tab. 6.







Separation Properties
All Fuel 
Throughput [tML]
60
feed buffer [tML]
66
Pu output  [tML]
6
Pu separation efficiency
0.99
Recycled Uranium [tML]
Unlimited
U separation efficiency
0.99
Waste [tML]
Unlimited
Tab. 6. Separation facilities core properties.


As in the calculation definition the only differences in the results can be observed in the behaviour of the reprocessed fuel, the UNF fuel waiting for reprocessing and the composition of the storage, as the reprocessing process directly impact them.
Fig. 8a. Annual Reprocessing Rate				Fig. 8b Fuel waiting for reprocessing

Fig. 8c. Storage Fuel composition

Indeed, as observed on Fig. 8a., the reprocessing follow the plutonium need of the FBR fuel fabrication. The UOX irradiated fuel only contain about only 1 % of plutonium, when FBR irradiated contain about 10%, explaining the sudden decrease of the  amount of fuel reprocessed when switching from UOX fuel reprocessed to FBR fuel.
Because of both the reprocessing priority and ?on demand? reprocess deployment, we can observed more fuel waiting for reprocessing, with a sudden consumption around 2125. We can also observe some of FBR fuel waiting for reprocessing, which decrease slowly, showing a well designed FBR deployment schedule...
\section{Summary}
This study have shown the capacity of CYCLUS to properly simulate a transition such as EG01 to EG23 transition.
The main observable differences are in the reprocessing and the storage of the used fuel, where it was not clear how it was managed in VISION. So even if I have been able to reproduce the overall calculation perform with B. Feng, some uncertainty remain on the exact signification of some data, like difference between UNF waiting for reprocessing and storage.
Despite CYCLUS is not able now to handle exact on demand behaviour, it is possible to deploy the different facilities with limited capacities following the material demand to mimic a on demand behaviour, as shown in the second calculation.

We can also observe some small difference in the pattern of fuel loading ( and almost all the reactor fuel metric), this comes from the way to model the batch in CYCLUS where in VISION all entities are managed with incoming and outgoing continuous flux.



%----------------------------------------------------------------------------------------
%	BIBLIOGRAPHY
%----------------------------------------------------------------------------------------

\bibliographystyle{unsrt}

\bibliography{}

%----------------------------------------------------------------------------------------


\end{document}