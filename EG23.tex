\documentclass[12pt]{article}

% Any percent sign marks a comment to the end of the line

% Every latex document starts with a documentclass declaration like this
% The option dvips allows for graphics, 12pt is the font size, and article
%   is the style




% array/table
\usepackage{array}
\usepackage{multirow}

% figure
\usepackage[pdftex]{graphicx}
\usepackage{epsfig}
\usepackage[hang]{subfigure}
\usepackage[small,bf]{caption}
\usepackage{amsmath}
\usepackage{enumitem}
\usepackage{tabularx}
\usepackage{ctable}


\usepackage{authblk}
\renewcommand\Authands{ and }

%clickable ref
\usepackage[backref,pagebackref,naturalnames=true,colorlinks]{hyperref}



\setlength{\oddsidemargin}{0.25in}
\setlength{\textwidth}{6.5in}
\setlength{\topmargin}{0in}
\setlength{\textheight}{8.5in}



%----------------------------------------------------------------------------------------
%	DOCUMENT INFORMATION
%----------------------------------------------------------------------------------------

\title{EG01-23: CYCLUS Calculation Report}


\author[1]{B. MOUGINOT\thanks{\href{mailto:mouginot@wisc.edu}{mouginot@wisc.edu}}}
\author[1]{P.P.H. WILSON \thanks{\href{mailto:paul.wilson@wisc.edu}{paul.wilson@wisc.edu}}}
\author[1]{R. CARLSEN \thanks{\href{mailto:carlsen@wisc.edu}{carlsen@wisc.edu}}}
\affil[1]{University of Wisconsin-Madison, Department of Engineering Physics, CNERG group}


\date{\today}

\begin{document}
\maketitle

\section{Intro/specification}
The goal of this study is to model the transition
from once-through PWRs (EG01) to full recycle FBRs
(EG23). It appears that the immediate transition
between the existing PWR fleet (EG01) and a full
FBR fleet (EG23) was not directly possible. In
order to have enough plutonium inventories to do
the transition, a few LWRs must be deployed before
starting to replace LWRs with FBRs. The deployment
of the new LWRs and the FBRs was calculated to:
\begin{itemize}
\item minimize the amount of plutonium in the cycle,
\item follow the energy production requested (Fig\ref{fig:nrg}).
\end{itemize}
The deployment schedule (Fig.\ref{fig:deploy}) has not been
calculated using CYCLUS, it has been taken from B.
Feng calculation and implemented in CYCLUS (with
some approximation which will be detailed further
in the document).

\begin{figure}[h!]
\centering
\subfigure[Deployment schedule\label{fig:deploy}]     {\epsfig{figure=img/CapacityStarted,width=0.48\textwidth}}
\subfigure[Electricity produced\label{fig:nrg}]		{\epsfig{figure=img/ElectricityGenerated,width=0.48\textwidth}}
\caption{Deployment schedule and corresponding
electricity produced by the different reactors
where R1(blue): existing PWR, R2(green): new
builded PWR, R3(red): high FBR, R4(purple): low
FBR.\label{fig:deployment} }
\end{figure}


\pagebreak
\section{Prototype configuration: what \& why}
This calculation is based on previous calculation
operated by R. Carlsen, which can be found there.
From this study, the configuration and deployment
schedule of the separation facilities for PWR used
fuel has been taken. The rest of the study has
been adapted from it, to match B. Feng DYMOND
calculation \cite{B.Feng_calculation}.

As Cyclus does not yet include direct support for
on-demand processing, the separation and the
fabrication have been modeled as follows to
approximate on-demand separation, thus minimizing
inventories of separated plutonium.
The Cyclus archetypes have a greedy behavior, they
request and process as much material as they can.
Limitations can be set by fixing the amount of
feeding buffer, of the throughput and output
buffer.
This work aims to show the capabilities to
reproduce EG01-EG23 calculation using Cyclus, and
presents two illustrations of what can be done
using CYCLUS.

The first one correspond the modeling
of a growing MOX fuel fabrication capacity
following the MOX fuel requirement.


The second one improve the first calculation and
models also a grows of fuel separation capacities
following the demand grows.

In both the reactor characteristic (core
properties, fuel recipes, deployment schedule) are
the same.

\subsection{First Case}
\subsubsection{Reactors}
After the PWR/FBR transition, new FBR reactor are
deployed with a lower breeding ratio, in order to
minimize the plutonium inventory. The properties
of the different reactor core used for the
simulation have been summarized in
Tab.\ref{tab:reactor}.\\
\begin{table}[h!]
\centering
\begin{tabularx}{350pt}{lXXX}
\hline
Core Properties       &	PWR     &	FBR 1   &	FBR 2     \\
\hline
Rated Power, [MWe]    &	1000		&	400     &	400       \\
Thermal Efficiency    &	N.A.$^1$	&	0.4     &	0.4       \\
Capacity Factor       &	0.90		&	0.90		&	0.90      \\
Number of batches     &	6       &	5       &	3         \\
Cycle length, [month] &	9       &	13      &	18        \\
Core Inventory, [tHM] &	14.784*6&	7.524*5	&	11.466*3  \\
\hline
\end{tabularx}
\caption{Reactor core properties.}
\label{tab:reactor}
\footnotesize{$^1$ I don't have this number, but as a recipe reactor are used, it is not required}
\end{table}

Because of the 1 month timestep of
CYCLUS, some liberty has been taken with the
couple Batch-Quantity/Cycle-length in order to
match as closely as possible with the specified
quantities [REF-F.Bo].\\
For the FBR 1, the cycle lenght is supposed to be
5.44 y (for discharge burnup of 37.62 GWd/t), 5.41
 y has been used (5 * 13 months). For FBR 2, 4.5y
 has been used (3 * 18 months) where 4.44 was
 expected.\\
\textit{Since Cyclus have a 1 month time step, the
burnup precision can be easly determined as :
\begin{equation}
  \Delta BU = \frac{P_{th} \times \Delta t}{M_{r}},
\end{equation}
where $P_{th}$ is the thermal power of the reactor
, $M_{r}$ its mass in heavy metal, $\Delta BU$ the
burnup achievable precision and $\Delta t$ the
time step size.}\\
In this simulation the $\Delta BU$ are for the FBR
1 \& 2 respectively: $0.73$ and $0.79~GWd/t$.
This does not affect the composition of the
discharge fuel as Cyclus does not perform burnup
evolution calculation, this just affects the
masses amount of discharged fuel wich a relativ
error of:
\begin{equation}
  \frac{\Delta M}{M} = \frac{\Delta BU}{BU}.
\end{equation}

All PWR reactors use enriched uranium fuel. The
FBR used MOX fuel with different plutonium
enrichments. The input/output fuel recipe are
summarized in the Tab.\ref{tab:reactor_fuel}
(from \cite{B.Feng_calculation}).

\begin{table}[h!]
\centering
\begin{tabular}{lllll}
\hline
\multicolumn{2}{c}{Reactor}			&	PWR [$\%w$]	&	FBR fuel 1  [$\%w$]	&	FBR fuel 2  [$\%w$] 	\\
\hline
\multirow{3}{*} {In recipe}	&	235U	&	95.8			&	0				&	0				\\
&	238U	&	4.2			&	92.36			&	91.466			\\
&	Pu		&	0			&	7.64				&	8.534			\\
\hline
\multirow{5}{*} {Out recipe}&	235U	&	0.8			&	0				&	0				\\
&	238U/U	&	92.68		&	85.99			&	86.025			\\
&	Pu		&	1.2			&	9.02				&	9.596			\\
&	MA		&	0.11			&	0.13				&	0.107			\\
&	FP		&	5.21			&	4.86				&	4.272			\\
\hline
\end{tabular}
\caption{Input/Output Fuel composition recipe for
the different reactors. Note that for the FBR
reactor fuel no isotopic distinctions have been
made and U in FBR should be considered depleted
uranium in the input recipes, the uranium isotopic
change in the output recipes have not been
investigated in this work.  }
\label{tab:reactor_fuel}
\end{table}

\subsubsection{Cooling/Storage}
After Irradiation, all fuel are supposed to be
cooled for 84 months and then be available for
reprocessing.\\
The Cycacore Storage allows to define a minimum
residence time for each incoming material.
Nevertheless, in order to simplify the
differenciation between material in cooling and
material available for rerocessing, 2 storage have
been defined for each types of fuels. On dedicated
to the cooling, with a residence time of 84 months
, and one dedicated to the lonf term storage with
no residence time.
When the first one as obviously be used to fill
the Fuel in cooling data, the second one have been
use to fill the "UNF waiting for reprocessing"
data as well as the "Ressource in storage" (of the
output spreadsheet). This might not be the correct
 data attempted in "Ressource in storage"...

\subsubsection{Separation}

The LWR fuel separation is handled by three
identical separation facilities, two deployed in
2030 and one in 2040. The FBR separation
facilities have a very large separation capacity,
in the first case, we have tried to limit the
separation flux using the fuel fabrication
facilities: since the fabrication if full as well
as the separation, it will not request any more
plutonium, so the separation will stop the
separation...

\begin{table}[h!]
\centering
\begin{tabular}{lllll}
\hline
Separation Properties	&	PWR		&	FBR 1/2	\\
\hline
Throughput [tML]		&	83.3333	&	5000		\\
feed buffer [tML]		&	107.537	&	5000		\\
Pu output  [tML]		&	Unlimited	&	5000		\\
Pu separation efficiency	&	0.99		&	0.99		\\
Recycled Uranium [tML]	&	Unlimited	&	Unlimited	\\
U separation efficiency	&	0.99		&	0.99		\\
Waste [tML]			&	Unlimited	&	Unlimited	\\
\hline
\end{tabular}
\caption{Separation facilities core properties. }
\label{tab:separation_1}
\end{table}

\subsubsection{Fuel Fabrication}
The UOX fuel fabrication is handled by one
enrichment facility, the properties of this
enrichment facilities are summed up in
Tab.\ref{tab:enrich_1}.

\begin{table}[h!]
\centering
\begin{tabular}{lllll}
\hline
Enrichment Properties	&	UOX		\\
\hline
Throughput [tML]		&	Unlimited	\\
swu capacity [tML]		&	1e97		\\
tails assay  			&	0.0025	\\
Initial feed [tML]		&	Unlimited	\\
\hline
\end{tabular}
\caption{Enrichment facilities properties. }
\label{tab:enrich_1}
\end{table}

The FBR fuel fabrication are suppose to handle and
been deployed as a rate of 1 for 10 FBR reactors.
Since the fuel composition and annual flux
slightly change between FBR 1 and 2,, the
specifications between fabrication are, as well,
slightly differents... The detail of all fuel
fabrication facilities characteristics can be
found in Tab.\ref{tab:fuelfab_1}.

\begin{table}[h!]
\centering
\begin{tabular}{lllll}
\hline
Fuel Fab Properties	&	FBR 1	&	FBR 2	\\
\hline
Throughput [tML]	&	75.240	&	76.440	\\
depleted buffer [tML]	&	69.492	&	69.912	\\
Pu buffer  [tML]		&	5.748	&	5.856	\\
\hline
\end{tabular}
\caption{Fuel fabrication facilities properties.}
\label{tab:fuelfab_1}
\end{table}


\subsubsection{Results}
All fuel loading metrics
(Fig.\ref{fig:RessourceUsed}) are the same or very
close to the DYMOND simulations. Nevertheless one
can observe some fluctuation on the annual fuel
loading and the SWU requirement which are
consistent with the batching of the initially
deployed reactors which are synchronized.

\begin{figure}[h!]
\centering
\subfigure[Ressources Mined]			{\epsfig{figure=img/RessourceMined,width=0.48\textwidth}}
\subfigure[SWU Requirement]			{\epsfig{figure=img/SWURequierment,width=0.48\textwidth}}
\subfigure[Annual Fuel Loading Rate]	{\epsfig{figure=img/AnnualFuelLoading,width=0.48\textwidth}}
\caption{TBD.\label{fig:RessourceUsed} }
\end{figure}

The generated power and the deployment schedule
(Fig. \ref{})  match perfectly the one of the
DYMOND simulations. The sudden drop in 2210 is
due to a lack of data after 2210: no new reactor
have been started after this date.\\

\begin{figure}[h!]
\centering
\subfigure[Deployment schedule]	{\epsfig{figure=img/CapacityStarted,width=0.48\textwidth}}
\subfigure[Electricity produced ]		{\epsfig{figure=img/ElectricityGenerated,width=0.48\textwidth}}
\caption{Deployment schedule and corresponding
electricity produced by the different reactors
where R1(blue): existing PWR, R2(green): new
builded PWR, R3(red): high FBR, R4(purple): low
FBR.\label{fig:deployment_bis} }
\end{figure}


The other curves are as expected. Indeed, the
annual reprocessing rate correspond to the
deployment :  for the UOX fuel start the
reprocessing in 2010 with 2/3 of the capacity, and
then an increase of 1/3 in 2030, and then a
constant reprocessing rate (UOX1 + UOX2) until the
end of the production of used UOX (all PWR
decommissioning).


\begin{figure}[h!]
\centering
\includegraphics[width=0.62\textwidth]	{img/AnnualReprocessingRate_1}
\caption{Annual Reprocessing Rate.}
\label{fig:reprocessing_1}
\end{figure}


For the MOX fuel produced by the 2 different FBR
types, it follows the loading of fresh fuel, with
some shift : the reprocessing facility are greedy:
they reprocess all the used fuel available. In
consequence there is no FBR used fuel in storage
they are directly reprocess after the cooling.


\begin{figure}[h!]
\centering
\subfigure[Used Fuel in Cooling]		{\epsfig{figure=img/usedFuelInCooling,width=0.48\textwidth}}
\subfigure[Fuel waiting for reprocessing]	{\epsfig{figure=img/UNFWaitingReprocessing_1,width=0.48\textwidth}}
\caption{TBD.\label{fig:cool_reprocc} }
\end{figure}


Concerning the fuel in storage, as almost all FBR
used fuel are directly reprocess after cooling,
all quantity barely stop increasing after all the
UOX fuel have been reprocess.

\begin{figure}[h!]
\centering
\includegraphics[width=0.62\textwidth]{img/FuelInStorage_1}
\caption{Storage Fuel composition.}
\label{fig:storagecompo_1}
\end{figure}

It appears that the capacity of the FBR fuel
reprocessing are not enough to reprocess the full
income of MOX fuel until 2195: this explain the
some appearance of MOX fuel in storage waiting for
reprocessing. With the deployment of the low
breeder FBR, the high breeder reactor are
decommissioned decreasing the incoming irradiated
MOX fuel,  the reprocessing capacity is again
higher than the sued MOX (type 1) production...


As shown in the figure \ref{fig:FC_Z} \&
\ref{fig:WR_Z}, the quantity of plutonium in the
storage follow the variation of the amount of MOX1
fuel in the storage.

\begin{figure}[h!]
\centering
\subfigure[Storage Fuel composition (zoomed)\label{fig:FC_Z}]		{\epsfig{figure=img/FuelInStorage_1_zoom,width=0.48\textwidth}}
\subfigure[Fuel waiting for reprocessing (zoomed)\label{fig:WR_Z}]	{\epsfig{figure=img/UNFWaitingReprocessing_1_zoom,width=0.48\textwidth}}
\caption{TBD.\label{fig:FC_WR_zoom} }
\end{figure}

\subsection{Second Case}
In parallel of the previous calculation, I have
tried to perform a calculation, where the
separation facilities have limited capacities and
are deployed accordingly to the plutonium
requirement for MOX fuel fabrication. This
deployment schedule is not perfect, but it
provides a good idea to how should evolve such
kind of calculation...
The only difference with the previous calculation
is the way to define the reprocessing.

For this calculation, all fuel are reprocessed by
only one kind of reprocessing facility. The fuel
input preferences has been set to 3 to all
irradiated UOX fuel, 2 for MOX fuel irradiated in
high breeder, and 3 for the low breeder fuel
(higher the number is higher the preferences is).
The characteristic of those reprocessing
facilities are summed up in
Tab.\ref{tab:fuelfab_2}.

\begin{table}[h!]
\centering
\begin{tabular}{ll}
\hline
Separation Properties	&	All Fuel	\\
\hline
Throughput [tML]		&	60		\\
feed buffer [tML]		&	66		\\
Pu output  [tML]		&	6		\\
Pu separation efficiency	&	0.99		\\
Recycled Uranium [tML]	&	Unlimited	\\
U separation efficiency	&	0.99		\\
Waste [tML]			&	Unlimited	\\
\hline
\end{tabular}
\caption{Separation facilities core properties.}
\label{tab:fuelfab_2}
\end{table}


As in the calculation definition the only
differences in the results can be observed in the
behavior of the reprocessed fuel, the UNF fuel
waiting for reprocessing and the composition of
the storage, as the reprocessing process directly
impact them.

\begin{figure}[h!]
\centering
\subfigure[Annual Reprocessing Rate]			{\epsfig{figure=img/AnnualReprocessingRate_2,width=0.48\textwidth}}
\subfigure[Fuel waiting for reprocessing]			{\epsfig{figure=img/UNFWaitingReprocessing_2,width=0.48\textwidth}}
\subfigure[Storage Fuel composition]	{\epsfig{figure=img/FuelInStorage_2,width=0.48\textwidth}}
\caption{TBD.\label{fig:ARR_FWR_SFC_2} }
\end{figure}

Indeed, as observed on Fig. 8a., the reprocessing
follow the plutonium need of the FBR fuel
fabrication. The UOX irradiated fuel only contain
about only 1\% of plutonium, when FBR irradiated
contain about 10\%, explaining the sudden decrease
of the  amount of fuel reprocessed when switching
from UOX fuel reprocessed to FBR fuel.
Because of both the reprocessing priority and "on
demand" reprocess deployment, we can observed more
fuel waiting for reprocessing, with a sudden
consumption around 2125. We can also observe some
of FBR fuel waiting for reprocessing, which
decrease slowly, showing a well designed FBR
deployment schedule...

\section{Summary}
This study have shown the capacity of CYCLUS to
properly simulate a transition such as EG01 to
EG23 transition.\\
The main observable differences are in the
reprocessing and the storage of the used fuel,
where it was not clear how it was managed in
DYMOND. So even if I have been able to reproduce
the overall calculation perform with B. Feng, some
uncertainty remain on the exact signification of
some data, like difference between UNF waiting for
reprocessing and storage.\\
Despite CYCLUS is not able now to handle exact on
demand behavior, it is possible to deploy the
different facilities with limited capacities
following the material demand to mimic a on demand
behavior, as shown in the second calculation.

We can also observe some small difference in the
pattern of fuel loading ( and almost all the
reactor fuel metric), this comes from the way to
model the batch in CYCLUS where in DYMOND all
entities are managed with incoming and outgoing
continuous flux.


%----------------------------------------------------------------------------------------
%	BIBLIOGRAPHY
%----------------------------------------------------------------------------------------

\bibliographystyle{unsrt}

\bibliography{}

%----------------------------------------------------------------------------------------

\end{document}
