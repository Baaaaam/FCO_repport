\documentclass[12pt]{article}

% Any percent sign marks a comment to the end of the line

% Every latex document starts with a documentclass declaration like this
% The option dvips allows for graphics, 12pt is the font size, and article
%   is the style




% array/table
\usepackage{array}
\usepackage{multirow}

% figure
\usepackage[pdftex]{graphicx}
\usepackage{epsfig}
\usepackage[hang]{subfigure}
\usepackage[small,bf]{caption}
\usepackage{amsmath}
\usepackage{enumitem}
\usepackage{tabularx}
\usepackage{ctable}


\usepackage{authblk}
\renewcommand\Authands{ and }

%clickable ref
\usepackage[backref,pagebackref,naturalnames=true,colorlinks]{hyperref}



\setlength{\oddsidemargin}{0.25in}
\setlength{\textwidth}{6.5in}
\setlength{\topmargin}{0in}
\setlength{\textheight}{8.5in}



%----------------------------------------------------------------------------------------
%	DOCUMENT INFORMATION
%----------------------------------------------------------------------------------------

\title{EG01-EG23: Cyclus Results}


\author[1]{B. Mouginot\thanks{\href{mailto:mouginot@wisc.edu}{mouginot@wisc.edu}}}
\author[1]{P.P.H. Wilson \thanks{\href{mailto:paul.wilson@wisc.edu}{paul.wilson@wisc.edu}}}
\author[1]{R. Carlsen \thanks{\href{mailto:rcarlsen@wisc.edu}{rcarlsen@wisc.edu}}}
\affil[1]{University of Wisconsin--Madison, Department of Engineering Physics, CNERG group}


\date{\today}

\setlength{\parindent}{0em}
\setlength{\parskip}{0.7em}

\begin{document}
\maketitle

\section{Intro/specification}

The goal of this study is to model the transition from a once-through light
water reactor (LWR) fuel cycle (i.e. EG01) to a sodium fast reactor (SFR) full
recycle case (i.e. EG23).   The LWR and SFR deployments were selected to
accomplish three primary goals:

\begin{itemize}
    \item Follow the 1\% annual electricity production curve (Figure \ref{fig:nrg}).
    \item Minimize the amount of sitting separated plutonium.
    \item Transition as quickly as possible given the other goals.
\end{itemize}

\section{Scenario Description}

By default, Cyclus facilities exhibit greedy behavior, they request and
process as much material as they can given their inventory capacities and
process limits.  As Cyclus does not yet include direct support for on-demand
processing, some special accommodations were made to maintain the desired low
inventory of separated Pu.  In order to address this in parts, a series of two
simulations were run:

\begin{itemize}

    \item \textbf{Case 1}: A simulation with fuel fabrication deployments
        matching capacity for SFR fresh fuel demand over time, but with
        unrestricted, greedy spent fuel separations.

    \item \textbf{Case 2}: A simulation with separations throughput changing
        over time. Results from the first simulation were used to calculate
        the required separations deployments over time in order to approximate
        on-demand separations.

\end{itemize}

Scenario details for cases 1 and 2 are identical except for those noted in
Section \ref{sec:case2}.

Scenario details are based on previous work done by R. Carlsen, which
can be found there (TODO: where?).  From this study, the configuration and
deployment schedule of the separation facilities for PWR used fuel have been
taken. The rest of the study has been adapted from it, to match B. Feng DYMOND
analysis \cite{B.Feng_calculation}.

The deployment schedule (Figure \ref{fig:deploy}) has not been calculated
using Cyclus, it has been taken from B. Feng calculation and implemented in
Cyclus (with some approximation which will be detailed further in the
document).

\begin{figure}[h!]
    \centering
    \subfigure[Deployment schedule\label{fig:deploy}]     {\epsfig{figure=img/CapacityStarted,width=0.48\textwidth}}
    \subfigure[Electricity produced\label{fig:nrg}]		{\epsfig{figure=img/ElectricityGenerated,width=0.48\textwidth}}
    \caption{
        Deployment schedule and corresponding electricity produced by the
        different reactors where R1(blue): existing PWR, R2(green): new
        builded PWR, R3(red): high SFR, R4(purple): low
        SFR.\label{fig:deployment}
    }
\end{figure}

The following subsections describe the configuration of the primary facility
types (i.e. Cyclus prototypes) used in this analysis.

\subsection{Reactors}

After the PWR/SFR transition, new SFR reactors are deployed with a lower
breeding ratio in order to minimize the plutonium inventory. The properties of
the different reactors cores used for the simulation have been summarized in
Table \ref{tab:reactor}.

\begin{table}[h!]
\centering
\begin{tabularx}{350pt}{lXXX}
\hline
Core Properties       &	PWR     &	SFR 1   &	SFR 2     \\
\hline
Rated Power, [MWe]    &	1000		&	400     &	400       \\
Thermal Efficiency    &	N.A.$^1$	&	0.4     &	0.4       \\
Capacity Factor       &	0.90		&	0.90		&	0.90      \\
Number of batches     &	6       &	5       &	3         \\
Cycle length, [month] &	9       &	13      &	18        \\
Core Inventory, [tHM] &	14.784*6&	7.524*5	&	11.466*3  \\
\hline
\end{tabularx}
\caption{Reactor core properties.}
\label{tab:reactor}
\footnotesize{$^1$ I don't have this number, but as a recipe reactor is used, it is not required.}
\end{table}

Because of the 1 month timestep of Cyclus, some liberties have been taken with
the couple Batch-Quantity/Cycle-length in order to match as closely as
possible with the specified quantities \cite{REF-F.Bo}.

For the SFR 1, the cycle lenght is supposed to be 5.44 y (for discharge burnup
of 37.62 GWd/t), 5.41 y has been used (5 * 13 months). For SFR 2, 4.5y has
been used (3 * 18 months) where 4.44 was expected.

As Cyclus has a 1 month time step, the
burnup precision can be easly determined as:

\begin{equation}
  \Delta BU = \frac{P_{th} \times \Delta t}{M_{r}},
\end{equation}

\noindent where $P_{th}$ is the thermal power of the reactor, $M_{r}$ its mass
in heavy metal, $\Delta BU$ the burnup achievable precision and $\Delta t$ the
time step size.

In this simulation the $\Delta BU$ are for the SFR 1 \& 2 respectively: $0.73$
and $0.79~GWd/t$.  This does not affect the composition of the discharge fuel
as Cyclus does not perform burnup evolution calculation, this just affects the
masses amount of discharged fuel with a relative error of:

\begin{equation}
  \frac{\Delta M}{M} = \frac{\Delta BU}{BU}.
\end{equation}

All PWR reactors use enriched uranium fuel. The SFR used MOX fuel with
different plutonium enrichments. The input/output fuel recipe are summarized
in the Table \ref{tab:reactor_fuel} from \cite{B.Feng_calculation}.

\begin{table}[h!]
    \centering
    \begin{tabular}{lllll}
    \hline
    \multicolumn{2}{c}{Reactor}			&	PWR [$\%w$]	&	SFR fuel 1  [$\%w$]	&	SFR fuel 2  [$\%w$] 	\\
    \hline
    \multirow{3}{*} {In recipe}	&	235U	&	95.8			&	0				&	0				\\
    &	238U	&	4.2			&	92.36			&	91.466			\\
    &	Pu		&	0			&	7.64				&	8.534			\\
    \hline
    \multirow{5}{*} {Out recipe}&	235U	&	0.8			&	0				&	0				\\
    &	238U/U	&	92.68		&	85.99			&	86.025			\\
    &	Pu		&	1.2			&	9.02				&	9.596			\\
    &	MA		&	0.11			&	0.13				&	0.107			\\
    &	FP		&	5.21			&	4.86				&	4.272			\\
    \hline
    \end{tabular}

    \caption{
        Input/Output Fuel composition recipe for the different reactors. Note that
        for the SFR reactor fuel no isotopic distinctions have been made and U in
        SFR should be considered depleted uranium in the input recipes, the
        uranium isotopic changes in the output recipes have not been investigated
        in this work.
    }

    \label{tab:reactor_fuel}
\end{table}

\subsection{Cooling/Storage}

After irradiation, all fuel are supposed to be cooled for 84 months and then
be available for reprocessing.

The Cycacore Storage allows to define a minimum residence time for each
incoming material.  Nevertheless, in order to simplify the differenciation
between material in cooling and material available for reprocessing, 2
storages have been defined for each types of fuels. One dedicated to the
cooling, with a residence time of 84 months, and one dedicated to the long
term storage with no residence time.  When the first one has obviously been
used to fill the Fuel in cooling data, the second one has been used to fill
the "UNF waiting for reprocessing" data as well as the "Ressource in storage"
(of the output spreadsheet). This might not be the correct data attempted in
"Ressource in storage".

\subsection{Separation}

The LWR fuel separation is handled by three identical separation facilities,
two deployed in 2030 and one in 2040. The SFR separation facilities have a
very large separation capacity, in the first case, we have tried to limit the
separation flux using the fuel fabrication facilities: as the fabrication is
full as well as the separation, it will not request any more plutonium, so the
separation will stop.

\begin{table}[h!]
    \centering
    \begin{tabular}{lllll}
    \hline
    Separation Properties	&	PWR		&	SFR 1/2	\\
    \hline
    Throughput [tML]		&	83.3333	&	5000		\\
    feed buffer [tML]		&	107.537	&	5000		\\
    Pu output  [tML]		&	Unlimited	&	5000		\\
    Pu separation efficiency	&	0.99		&	0.99		\\
    Recycled Uranium [tML]	&	Unlimited	&	Unlimited	\\
    U separation efficiency	&	0.99		&	0.99		\\
    Waste [tML]			&	Unlimited	&	Unlimited	\\
    \hline
    \end{tabular}
    \caption{Separation facilities core properties. }
    \label{tab:separation_1}
\end{table}

\subsection{Fuel Fabrication}

The UOX fuel fabrication is handled by one enrichment facility, the properties
of this enrichment facilities are summed up in Table \ref{tab:enrich_1}.

\begin{table}[h!]
    \centering
    \begin{tabular}{lllll}
    \hline
    Enrichment Properties	&	UOX		\\
    \hline
    Throughput [tML]		&	Unlimited	\\
    swu capacity [tML]		&	1e97		\\
    tails assay  			&	0.0025	\\
    Initial feed [tML]		&	Unlimited	\\
    \hline
    \end{tabular}
    \caption{Enrichment facilities properties. }
    \label{tab:enrich_1}
\end{table}

The SFR fuel fabrication are supposed to handle and be deployed as a rate of 1
for 10 SFR reactors.  As the fuel composition and annual flux slightly change
between SFR 1 and 2, the specifications between fabrication are, as well,
slightly different. The details of all fuel fabrication facilities
characteristics can be found in Table \ref{tab:fuelfab_1}.

\begin{table}[h!]
    \centering
    \begin{tabular}{lllll}
    \hline
    Fuel Fab Properties	&	SFR 1	&	SFR 2	\\
    \hline
    Throughput [tML]	&	75.240	&	76.440	\\
    depleted buffer [tML]	&	69.492	&	69.912	\\
    Pu buffer  [tML]		&	5.748	&	5.856	\\
    \hline
    \end{tabular}
    \caption{Fuel fabrication facilities properties.}
    \label{tab:fuelfab_1}
\end{table}

\section{On-demand Separations: Case 2 Differences}
\label{sec:case2}

The only difference with the Case 1 scenario is the way in
which reprocessing capacity is handled over time.  Modeling on-demand
separations requires deploying separations facilities according to plutonium
requirements for SFR fuel fabrication.  These Pu requirements were determined
from the Case 1 results. While the calculated deployment schedule does not
follow plutonium usage exactly, it is a good rough aproximation. 

For this calculation, all fuel are reprocessed by only one type of
reprocessing facility (i.e. one Cyclus prototype). The fuel input preferences
have been set to 3 for all irradiated UOX fuel, 2 for higher breeding ratio
SFR spent fuel, and 3 for the lower breeding ratio SFR spent fuel (a higher
number indicates a higher preference).  The characteristics of each of the
reprocessing facilities used in Case 2 are summarized in Table
\ref{tab:fuelfab_2}.

\begin{table}[h!]
    \centering
    \begin{tabular}{ll}
    \hline
    Separation Properties	&	All Fuel	\\
    \hline
    Throughput [tML]		&	60		\\
    feed buffer [tML]		&	66		\\
    Pu output  [tML]		&	6		\\
    Pu separation efficiency	&	0.99		\\
    Recycled Uranium [tML]	&	Unlimited	\\
    U separation efficiency	&	0.99		\\
    Waste [tML]			&	Unlimited	\\
    \hline
    \end{tabular}
    \caption{Separation facilities core properties.}
    \label{tab:fuelfab_2}
\end{table}

\section{Results}

\subsection{Case 1: Greedy Separations}

All fuel loading metrics (Figure \ref{fig:RessourceUsed}) are the same or very
close to the DYMOND simulations. Nevertheless one can observe some
fluctuations on the annual fuel loading and the SWU requirement which are
consistent with the batching of the initially deployed reactors which are
synchronized.

\begin{figure}[h!]
    \centering
    \subfigure[Ressources Mined]			{\epsfig{figure=img/RessourceMined,width=0.48\textwidth}}
    \subfigure[SWU Requirement]			{\epsfig{figure=img/SWURequierment,width=0.48\textwidth}}
    \subfigure[Annual Fuel Loading Rate]	{\epsfig{figure=img/AnnualFuelLoading,width=0.48\textwidth}}
    \caption{TBD.\label{fig:RessourceUsed} }
\end{figure}

The generated power and the deployment schedule (Figure \ref{})  match perfectly
the one of the DYMOND simulations. The sudden drop in 2210 is due to a lack of
data after 2210: no new reactor has been started after this date.

\begin{figure}[h!]
    \centering
    \subfigure[Deployment schedule]	{\epsfig{figure=img/CapacityStarted,width=0.48\textwidth}}
    \subfigure[Electricity produced ]		{\epsfig{figure=img/ElectricityGenerated,width=0.48\textwidth}}
    \caption{Deployment schedule and corresponding
    electricity produced by the different reactors
    where R1(blue): existing PWR, R2(green): new
    builded PWR, R3(red): high SFR, R4(purple): low
    SFR.\label{fig:deployment_bis} }
\end{figure}


The other curves are as expected. Indeed, the annual reprocessing rate
correspond to the deployment:  for the UOX fuel start the reprocessing in 2010
with 2/3 of the capacity, and then an increase of 1/3 in 2030, and then a
constant reprocessing rate (UOX1 + UOX2) until the end of the production of
used UOX (all PWR decommissioning).


\begin{figure}[h!]
    \centering
    \includegraphics[width=0.62\textwidth]	{img/AnnualReprocessingRate_1}
    \caption{Annual Reprocessing Rate.}
    \label{fig:reprocessing_1}
\end{figure}


For the MOX fuel produced by the 2 different SFR types, it follows the loading
of fresh fuel, with some differences. The reprocessing facilities are greedy
--- they reprocess all used fuel available. As a consequence there is no SFR
used fuel in storage because it is directly reprocessed after cooling.

\begin{figure}[h!]
    \centering
    \subfigure[Used Fuel in Cooling]		{\epsfig{figure=img/usedFuelInCooling,width=0.48\textwidth}}
    \subfigure[Fuel waiting for reprocessing]	{\epsfig{figure=img/UNFWaitingReprocessing_1,width=0.48\textwidth}}
    \caption{TBD.\label{fig:cool_reprocc} }
\end{figure}


Concerning the fuel in storage, as almost all SFR used fuel are directly
reprocess after cooling, all quantity barely stop increasing after all the UOX
fuel have been reprocess.

\begin{figure}[h!]
    \centering
    \includegraphics[width=0.62\textwidth]{img/FuelInStorage_1}
    \caption{Storage Fuel composition.}
    \label{fig:storagecompo_1}
\end{figure}

It appears that the capacity of the SFR fuel reprocessing are not enough to
reprocess the full income of MOX fuel until 2195. This explains the appearance
of MOX fuel in storage waiting for reprocessing. With the deployment of the
low breeder SFR, the high breeder reactors are decommissioned decreasing the
incoming irradiated MOX fuel,  the reprocessing capacity is again higher than
the sued MOX (type 1) production.


As shown in the Figures \ref{fig:FC_Z} \& \ref{fig:WR_Z}, the quantity of
plutonium in the storage follows the variation of the amount of MOX1 fuel in
the storage.

\begin{figure}[h!]
    \centering
    \subfigure[Storage Fuel composition (zoomed)\label{fig:FC_Z}]		{\epsfig{figure=img/FuelInStorage_1_zoom,width=0.48\textwidth}}
    \subfigure[Fuel waiting for reprocessing (zoomed)\label{fig:WR_Z}]	{\epsfig{figure=img/UNFWaitingReprocessing_1_zoom,width=0.48\textwidth}}
    \caption{TBD.\label{fig:FC_WR_zoom} }
\end{figure}

\subsection{Case 2: On-demand Separations}

The only differences in the results can be observed in the behavior of the
reprocessed fuel, the UNF fuel waiting for reprocessing and the composition of
the storage.

\begin{figure}[h!]
    \centering
    \subfigure[Annual Reprocessing Rate]			{\epsfig{figure=img/AnnualReprocessingRate_2,width=0.48\textwidth}}
    \subfigure[Fuel waiting for reprocessing]			{\epsfig{figure=img/UNFWaitingReprocessing_2,width=0.48\textwidth}}
    \subfigure[Storage Fuel composition]	{\epsfig{figure=img/FuelInStorage_2,width=0.48\textwidth}}
    \caption{TBD.\label{fig:ARR_FWR_SFC_2} }
\end{figure}

Indeed, as observed on Figure 8a., the reprocessing follows the plutonium need
of the SFR fuel fabrication. The UOX irradiated fuel only contains about only
1\% of plutonium, when SFR irradiated contains about 10\%, explaining the
sudden decrease of the  amount of fuel reprocessed when switching from UOX
fuel reprocessed to SFR fuel.  Because of both the reprocessing priority and
"on demand" reprocess deployment, we can observed more fuel waiting for
reprocessing, with a sudden consumption around 2125. We can also observe some
of SFR fuel waiting for reprocessing, which decrease slowly, showing a well
designed SFR deployment schedule.

\section{Summary}

This study has shown the capacity of Cyclus to properly simulate a transition
such as EG01 to EG23 transition.  The main observable differences are in the
reprocessing and the storage of the used fuel, where it was not clear how it
was managed in DYMOND. So even if I have been able to reproduce the overall
calculation perform with B. Feng, some uncertainties remain on the exact
signification of some datas, like difference between UNF waiting for
reprocessing and storage.  Despite Cyclus is not able now to handle exact on
demand behavior, it is possible to deploy the different facilities with
limited capacities following the material demand to mimic a "on demand"
behavior, as shown in the second calculation.  We can also observe some small
differences in the pattern of fuel loading (and almost all the reactor fuel
metric), this comes from the way to model the batch in Cyclus where in DYMOND
all entities are managed with incoming and outgoing continuous flux.

%----------------------------------------------------------------------------------------
%	BIBLIOGRAPHY
%----------------------------------------------------------------------------------------

\bibliographystyle{unsrt}

\bibliography{}

%----------------------------------------------------------------------------------------

\end{document}
