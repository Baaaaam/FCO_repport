\documentclass[12pt]{article}

% Any percent sign marks a comment to the end of the line

% Every latex document starts with a documentclass declaration like this
% The option dvips allows for graphics, 12pt is the font size, and article
%   is the style




% array/table
\usepackage{array}
\usepackage{multirow}

% figure
\usepackage[pdftex]{graphicx}
\usepackage{epsfig}
\usepackage[hang]{subfigure}
\usepackage[small,bf]{caption}
\usepackage{amsmath}
\usepackage{enumitem}
\usepackage{tabularx}
\usepackage{ctable}


\usepackage{authblk}
\renewcommand\Authands{ and }

%clickable ref
\usepackage[backref,pagebackref,naturalnames=true]{hyperref}



\setlength{\oddsidemargin}{0.25in}
\setlength{\textwidth}{6.5in}
\setlength{\topmargin}{0in}
\setlength{\textheight}{8.5in}



%----------------------------------------------------------------------------------------
%        DOCUMENT INFORMATION
%----------------------------------------------------------------------------------------

\title{}


\author[1]{B. Mouginot  \thanks{\href{mailto:mouginot@wisc.edu}{mouginot@wisc.edu}}}
\author[1]{P.P.H. Wilson\thanks{\href{mailto:paul.wilson@wisc.edu}{paul.wilson@wisc.edu}}}
\author[1]{R. Carlsen} 
\author[1]{A. Opotowsky} 
\affil[1]{University of Wisconsin--Madison, Department of Engineering Physics, CNERG group}


\date{\today}

\setlength{\parindent}{0em}
\setlength{\parskip}{0.7em}

\begin{document}
\maketitle

\section{Intro/specification}

This memo presents the modeling results of cases 1.1 to 1.3 of the EG29
scenario, defined by the use of a plutonium equivalent model for the fuel
fabrication. Case 1 of the EG29 calculation involves the modeling of a single
MOX-PWR at steady-state (see figure \ref{fig:puflow}).

\begin{figure}[h!]
  \centering
  \includegraphics[width=1\textwidth]  {img/puflow}
  \caption{Schematic of Pu mass flow for Case 1.x}
  \label{fig:puflow}
\end{figure}

Case 1 is subdivided into three sub-cases corresponding to calculations of
increasing fidelity: 
\begin{itemize}
  \item	1: without isotopic composition, 
  \item 2: with isotopic composition and no decay,
  \item 3: with isotopic composition and decay.
\end{itemize}


\section{Fixed mixing ratio vs Pu-equivalent theory}

The first part of this memo is dedicated to the comparison between fixed ratio
mix and Pu-equivalent theories for the fuel fabrication process.

Two variations on fuel-building were calculated for each sub-case (1.1 to 1.3).
The first calculation used a standard mixing fab (in Cyclus, the
``cycamore::mixer''). This mixed the E3” and the J1” streams using a constant mixing
ratio to build MOX fuel for the PWR, labeled “M”. The second calculation
used plutonium equivalent theory to determine the mixing fraction of each stream
to build the MOX fuel, labeled “W”.

\begin{figure}[h!]
  \centering
  \subfigure[J1'' stream plutonium flow\label{fig:J1s}] 
    {\epsfig{figure=img/C_1_x_J1s_pu_contribution,width=0.48\textwidth}}
  \subfigure[E3'' stream plutonium flow\label{fig:e3s}] 
    {\epsfig{figure=img/C_1_x_E3s_pu_contribution,width=0.48\textwidth}}
  \caption{Evolution of the plutonium content in the J1'' and  E3'' stream\label{fig:MW_flow} }
\end{figure}

The difference between J1'' and E3'' for all 6 calculations can be observed in
Figure \ref{fig:MW_flow}. Note that for this study one should only consider the time
after 15 as the calculation needs almost 12y to reach an equilibrium. 

For both streams (J1'' and E3''), the 2 cases without decay are similar in the 2
calculation methods (W and M). When decay is taken into account, one can
observe a small reduction of the plutonium from  J1'' stream as well as a small
increase of the plutonium from E3'' stream directly due to $^{241}$Pu decay: the E3''
is not affected much by the decay process as it contains mainly 239Pu, while the J1''
stream contains a higher $^{241}$Pu content and is very sensitive to decay.  The decay tends
to decrease the reactivity potential of the J1'' stream, which is compensated by
the increase of the E3'' stream in the mix.


\section{Model enrichment prediction}

The second part of this memo is dedicated to the comparison of the plutonium
fraction in the fresh PWR-MOX fuel as predicted by different category of models.
The models can be divided in two categories, one category for those able to mix
any stream with another (fixed mixing ratio and Pu-equivalent based models) and
another category that only allows a plutonium stream to be mixed into a uranium
stream.  

For the first category, the EG29 specification are applied as is (except for the
fuel fabrication). For the second category, the plutonium and uranium from J1''
and E3'' streams are separated.  Then the plutonium from J1'' and E3'' are mixed
according to the ratio provided in EG29 specifications. The model is used to
determine what proportion of this J1'' + E3'' plutonium stream is required to
build the PWR-MOX fuel to achieve the EG29 specifications (LWR, 50GWd/t, 1/3
batching).

The different models are:
\begin{itemize}
  \item fixed mixing ratio between J1'' and E3'' stream, (M)
  \item	Pu-equivalent based model to mix J1'' and E3'' stream, (W)
  \item	Neural Network (NN) trained with irradiation stopped at a $k_{\infty}$ of 1.01 (mean $k_{\infty}$ of all batches), 3 batches, (MLP)
  \item	NN with irradiation stopped at a mean a $k_{\infty}$ of 1.034, 3 batches, (MLP-STD)
  \item	NN with irradiation stopped at a mean a $k_{\infty}$ of 1.034, 4 batches. (MLP-STD-2)
\end{itemize}

In addition to the fuel fabrication, a CLASS model based on neural network has
been used to recalculate the proper evolution during the irradiation of the fuel
in the PWR reactor.

Note that some calculation based on neural network usage have been extended 200y
in order to allow the calculation to reach the equilibrium (since the initial
compositions are close to the fixed recipe calculation equilibrium).

\subsection{No decay}

\begin{figure}[h!]
  \centering
  \includegraphics[width=0.7\textwidth]  {img/C_1_2_MOX_pu_contribution}
  \caption{Evolution of the plutonium fraction in the MOX fuel loaded in PWR.
  These calculations do not include decay process.}
  \label{fig:pufrac_ND}
\end{figure}


In Figure \ref{fig:pufrac_ND}, one can observe the evolution of the plutonium
enrichment loaded in the PWR fuel depending on the model considered for the
fabrication.
The Neural Network model using a $k_{\infty}$ of 1.01 clearly underestimates the amount of
plutonium required compared to the other models: the required reactivity is
lower. We can observe the effect of batching (3 or 4) on the
neural network models using a $k_{\infty}$ of 1.034, which predict an initial enrichment
close to the one used in the more standard models: $7.5\%$ versus $7.8\%$. Those
models predict an enrichment slightly lower than the ones using the fixed mixing
fraction or Pu-equivalence.
The Pu-equivalent model tries to mix both streams to reach the composition of the fuel
used in the fixed mixing-fraction method. So it is expected to match or to be very
close.
For all fabrication models used, the behavior without decay is very close to
expected: we observe a small variation in the amount of plutonium: the
composition of the plutonium is constant, which allows the equilibrium to be
maintained. 


\subsection{Decay}

\begin{figure}[h!]
  \centering
  \includegraphics[width=0.7\textwidth]  {img/C_1_3_MOX_pu_contribution}
  \caption{Evolution of the plutonium fraction in the MOX fuel loaded in PWR.
  These calculations include the decay process.}
  \label{fig:puflow_D}
\end{figure}

When decay is taking into account, all models based on the neural network
increase the enrichment in plutonium by $2.5-3\%$. The $^{241}$Pu decays and
changes the final isotopic composition of the fuel.  The decay of $^{241}$Pu is
causing the degradation of the plutonium quality and the production of
$^{241}$Am (acting as a neutronics poison), both of those effects tend to
increase the amount of plutonium required to build the fuel.

In order to start the calculation, the fuel fabrication require a certain amount
of material. If those initial inventory are to big, it can influence the
calcultion: the inventory in the different storage are stakking and evolving
because of the decay process. In order to limit those effect when the decay
process is taken into account, all calculation have been tuned to reduce at the
minimum the amount of initial storage (which depends on the model used).

We can observe (see Figure \ref{fig:puflow_D} and Figure \ref{fig:pu_compo})
that as the $^{241}$Pu fraction decreases and the $^{239}$Pu fraction increases in
the used MOX composition, the amount of plutonium in the fresh fuel reaches an
equilibrium closer to the initial fixed ratio.

\begin{figure}[h!]
  \centering
  
  \subfigure[Evolution of the Pu composition in the fresh fuel using the MLP-STD
  fabrication model.\label{fig:pucompo_1}] 
      {\epsfig{figure=img/C_1_3_MLP-STD_MOX_pu_composition,width=0.48\textwidth}}
  \subfigure[Evolution of the Pu composition in the fresh fuel using the
    MLP-STD-2 fabrication model.\label{fig:pu_compo_2}] 
      {\epsfig{figure=img/C_1_3_MLP-STD-2_MOX_pu_composition,width=0.48\textwidth}}

  \caption{ Evolution of the isotopic composition of the plutonium in the fresh MOX fuel.
    \label{fig:pu_compo} } 
  \end{figure}

Even if the different model predictions fail to agree on a common equilibrium,
the point of the study is not determining which calculation is correct (in fact
both are probably correct), but to highlight the sensitivity of the equilibrium
state to the isotopic composition and the modeling choice (particularly for
thermal reactors): having overly strong constraints on the definition of the
fuel cycle may lead us on the wrong path\dots


In order to support this statement, 3 additional calculations have been performed
using the plutonium equivalence theory. In these calculations the fuel cycle is
the same as the one described as Case 1.3, but the initial inventory has been
increased by a factor of 2, 5 or 10.


\section{Plutonium stacking effect}

In the following calculation, the inventory of J1'' has been increased by a
factor 2, 5, or 10, in order to probe the effect of the of material stacking on
the J1 storage.

\begin{figure}[h!]
  \centering
  \includegraphics[width=0.6\textwidth]  {img/C_1_3_W_LII_x10_pu_composition}
  \caption{Evolution of the plutonium composition in the J1'' storage in the
  case where the initial inventory has been increased by a factor 10.}
  \label{fig:LII_compo_x10}
\end{figure}

As shown figure \ref{fig:LII_compo_x10}, the fraction of $^{241}$Pu drops from
about $12\%$ to less than $2\%$ in about 60y, while the $^{241}$Am rises from
$5\%$ to about $12\%$. Only the largest initial inventory is shown here (Figure
\ref{fig:LII_compo_x10}) as the effect is the strongest. All initial inventory
present the same behavior, only the ratio $^{241}$Pu versus $^{241}$Am change.
$^{241}$Pu and $^{241}$Am find their respective equilibria around $11\%$ and
$1\%$ for the normal calculation, $10\%$ and $2\%$ in the ``x2'' calculation
and, $3\%$ and $10\%$ in the ``x5'' case. This conversion of the $^{241}$Pu into
$^{241}$Am increases the average time the separated plutonium spends in storage
before being used to build fresh MOX fuel.

\begin{figure}[h!]
  \centering
  
  \subfigure[J1'' stream flow.\label{fig:E1_LII}] 
      {\epsfig{figure=img/C_1_3_W_LII_J1s.png,width=0.48\textwidth}}
  \subfigure[E3'' stream flow.\label{fig:E3_LII}] 
      {\epsfig{figure=img/C_1_3_W_LII_E3s.png,width=0.48\textwidth}}

  \caption{ Evolution of the J1'' and E3'' stream flow according to the size
    of initial inventory. \label{fig:LII} } 
  \end{figure}

As observed in Figure \ref{fig:LII}, the decay of the $^{241}$Pu has a direct
impact on the different contributions of the E3'' and J1'' streams. The model is
trying to balance the loss of reactivity of the J1'' stream by increasing the
amount of E3'' stream in the mix. The variation can be up to $10\%$ after 100y
in the worst case.

\section{Discussion}

In conclusion, this study has shown the impact of different ways to model
the MOX fuel fabrication, including whether or not decay is considered.
The Pu-equivalence theory reaches an equilibrium (in the mixing ratio of E3'' and
J1'' stream) very close to the fixed mixing ratio when the decay is not activated.
This was expected: the Pu-equivalent model tried to build a MOX fuel with the same
initial reactivity as the one used in the fixed mixing ratio calculation, so it
used the same mixing ratio\dots  

Nevertheless we can observe a variation of $20\%$ in the contribution of E3''
and J1'' streams  when taking the decay process into account. This variation is
a direct consequence of the decay of $^{241}$Pu producing $^{241}$Am, which
decreases the reactivity potential of J1'' ( loss of the $^{241}$Pu reactivity
and neutronic poisoning of the $^{241}$Am). The E3'' reactivity barely changes:
the $^{241}$Pu is negligible in E3'' composition.

The main difference between the calculation using the CLASS model and the other
calculation is the recalculation of the fuel evolution during irradiation. 

The calculation using the CLASS model based on a neural network supports this
observation. Without decay, this model predicts a constant plutonium enrichment
in the MOX fuel $6.6\%$, $7.4\%$ or $7.8\%$, depending on the model used, versus
about $8.4\%$ for the mixing ratio. This disagreement of a few percent in the
plutonium enrichment is a direct consequence of the modeling assumptions in the
different models.  But this common flat behavior highlights that the steady
state is well-described by all simulations when decay is not considered.  Even
if the CLASS model did not exactly agree on the plutonium enrichment required in
the fuel to get the correct reactivity properties, the output composition
recalculated by the CLASS model was very close to the out recipe used by the
other calculation because the equilibrium did not change.

When considering decay, all the CLASS models need about 180y to reach a proper
equilibrium. This is because the composition of the plutonium strongly impact
its enrichment in the fresh fuel (as shown in the Pu-equivalent case). The fresh
fuel composition will likewise strongly impact the plutonium composition at the
end of irradiation. 

Moreover, the last part of the study shows the impact of the stacked plutonium
on its composition and on fuel fabrication. This kind of material stacking could
the consequences for disruption if the material flow in a real cycle. Having
very strong constraints might cause such a disruption effect to be overlooked.
In order to mimic a potential material stacking, the same steady state
calculations have been perform three times, increasing the amount of initial
inventory required to start the calculation by a factor 2, 5 or 10. In all
cases, $^{241}$Pu stacking induces $^{241}$Am production which strongly impacts
the mixing ratio between E3'' and J1'' ( $5\%$ to $10\%$). This may have a
strong effect on the used fuel composition, specially with thermal reactors,if
the different cross sections are so sensitive to the composition of the fuel.



%----------------------------------------------------------------------------------------
%        BIBLIOGRAPHY
%----------------------------------------------------------------------------------------

\bibliographystyle{unsrt}

\bibliography{bib}

%----------------------------------------------------------------------------------------

\end{document}

